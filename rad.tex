\documentclass[titlepage]{article}
\usepackage{amssymb,amsmath,latexsym}
\usepackage[serbian]{babel}	
\usepackage[T1]{fontenc}
\newcommand{\qed}{\hfill $\Box$\vspace*{4mm}}
\newtheorem{df}{Definicija}
\begin{document}

	\title{\sc Geneteski Algoritmi i Problem Trgova\v{c}kog Putnika}

	\author{Daniel Sila\dj i\\
		Gimnazija "Jovan Jovanovi\'{c} Zmaj"\\
		Novi Sad}
	\maketitle

	\begin{abstract}
		Ovaj rad se bavi primenom genetskih algoritama u re\v{s}avanju problema trgova\v{c}kog putnika. \\
	\end{abstract}

	\section{Uvod}

    	\subsection{\v{S}ta su genetski algoritmi i \v{c}emu slu\v{z}e?}
    		Genetski algoritmi (u daljem tekstu GA) su porodica algoritama inspirisanih Darvinovom teorijom evolucije. Prvi radovi iz ove oblasti su nastali 60-tih godina pro\v{s}log veka, ali se u ve\'{c}ini izvora kao tvorac ove oblasti uzima John Holland. On je 1975. godine napisao knjigu "Adaptation in natural and artificial systems".
        GA imaju za cilj re\v{s}avanje problema kombinatorne optimizacije, tj problema u kojima se tra\v{z}i minimum ili maksimum neke funkcije . Po\v{s}to je prostor pretra\v{z}ivanja (skup re\v{s}enja) ponekad prevelik (i njegovo kompletno pretra\v{z}ivanje se ne mo\v{z}e izvr\v{s}iti u nekom doglednom vremenu), a optimalno re\v{s}enje nije neophodno (prihvata se i neko pribli\v{z}no, suboptimalno re\v{s}enje), mogu se koristiti genetski algoritmi. \\
      U GA, svako pjedina\v{c}no re\v{s}enje je predstavljeno jednom jedinkom, koja sadr\v{z}i gene, tj delove re\v{s}enja. Nad njima se vr\v{s}e operatori mutacije i ukr\v{s}tanja (kao mehanizam pretrage) i selekcije (usmerava algoritam ka perspektivnim delovima pretra\v{z}iva\v{c}kog prostora).

    	\subsection{Osnovni operatori u genetskim algoritmima}
            \subsubsection{Selekcija}
            U osnovnim crtama, princip rada selekcije je sli\v{c}an kao u stvarnom \v{z}ivotu: cilj je da se odabere genetski materijal koji ce se preneti u narednu generaciju. Kod ovog operatora je bitno sa\v{c}uvati raznovrsnost, kao i kvalitet genetskog materiala, ina\v{c}e \'{c}e se dobra re\v{s}enja mo\v{z}da zauvek izgubiti.
            \subsubsection{Ukr\v{s}tanje}
            Cilj ovog operatora je da se ukr\v{s}tanjem dva postoje\'{c}a re\v{s}enja dobiju nova, obi\v{c}no kvalitetnija re\v{s}enja.
            \subsubsection{Mutacija}
            Mutacijom se vra\'{c}a raznovrsnog genetskog materijala u populaciju. Ipak, prekomerna mutacija mo\v{z}e svesti algoritam na nasumi\v{c}nu pretragu, pa se zato uvodi verovatnoca mutacije, koja je uglavnom manja od 5\%.
    	\subsection{\v{S}ta su NP problemi?}
    	NP problemi su klasa problema za koje se veruje da ne postoji deterministi\v{c}ki polinomni algoritam koji ih re\v{s}ava. Ta\v{c}nije, NP problemi obuhvataju NP tvrde (kojima pripada i problem Trgova\v{c}kog putnika) i NP kompletne.
        \begin{description}
          \item[NP kompletni] problemi su ,,te\v{s}ki'' za re\v{s}avanje, ali se provera korektnosti njihovih re\v{s}enja izvodi u polinomnom vremenu
          \item[NP tvrdi] problemi su te\v{s}ki bar koliko i najte\v{z}i NP kompletni problemi, ali za proveravanje njihovih re\v{s}enja ne postoji algoritam polinomne slo\v{z}enosti
        \end{description}
        \subsection{\v{S}ta je problem trgova\v{c}kog putnika?}
        Problem trgova\v{c}kog putnika (engl. Travelling Salesman Problem - TSP) pripada klasi NP problema i defini\v{s}e se na slede\'{c}i na\v{c}in:
         \begin{df}
        Ako je dat kompletan te\v{z}inski graf, naci najkra\'{c}u putanju koja prolazi kroz sve \v{c}vorove grafa (Hamiltonovu putanju)
        \end{df}
        U literaturi se ponekad navodi i ,,Decision problem'', tj
        \begin{df}
        Ako je dat skup gradova i du\v{z}ina puta L, treba odrediti da li postoji Hamiltonov put kra\'{c}i od L?
        \end{df}
    \section{Kori\v{s}\'{c}ene metode}
        \subsection{Selekcija}
        U ovom radu su kori\v{s}cena su dva algoritma za selekciju:
        \begin{description}
          \item[Prosta, rulet selekcija] 
          \item[Selekcija zasnovana na rangu] sdf
        \end{description}

	\end{document} 